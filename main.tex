\documentclass[12pt]{article}
\usepackage{setspace,palatino,multirow}
\usepackage{amsmath, amsfonts, amssymb, amsthm}
\usepackage{latexsym}
\usepackage{graphicx}
\usepackage{harvard}
\usepackage{longtable}
\usepackage{booktabs}
\usepackage{appendix}
\usepackage{listings}
\usepackage{dcolumn}

\usepackage{graphicx}
\usepackage{float}
\usepackage[usenames,dvipsnames]{color}
\usepackage[margin=1in]{geometry}
\usepackage{pdflscape}
\usepackage[flushleft]{threeparttable}
\usepackage[hang,flushmargin]{footmisc}
\usepackage{color}
\usepackage{indentfirst}
\usepackage{cite}

\usepackage[affil-it]{authblk} 
\usepackage{etoolbox}
\usepackage{lmodern}



\makeatletter
\patchcmd{\@maketitle}{\LARGE \@title}{\fontsize{16}{19.2}\selectfont\@title}{}{}
\makeatother

\setlength{\parskip}{1em}
\renewcommand\Authfont{\fontsize{14}{14.4}\selectfont}
\renewcommand\Affilfont{\fontsize{12}{14.4}\selectfont}
\renewcommand{\baselinestretch}{1.3}
\usepackage[utf8]{inputenc}


\title{Overview of The Clean Coder}
\author{Akash Singh}
\affil{School of Computer Science, McGill University, Montreal, Canada}
\setcounter{MaxMatrixCols}{10}
% \doublespacing
\date{}
\begin{document}
\maketitle
% inline
{\Large \textbf{Clean Code and its Importance}\par}


Have you ever been disrupted by bad code? If you are a programmer with any level of experience, you would have come across such situations. There is a name for it: \textit{wading}. Obviously, a working mess is better than nothing. Have you ever thought of this, "I will fix it later." :) [I am feeling guilty!]

"Later equals never."

Bad coding practice leads to messy code. As the mess builds, the productivity of the team continues to decrease, asymptotically approaching zero. This leads to addition of more staff in the team in hopes of increasing productivity. But that new staff is not versed in the design of the system. They don’t know the difference between a change that matches the design intent and a change that thwarts the design intent. Furthermore, they, and everyone else on the team, are under horrific pressure to increase productivity. So they all make more and more messes, driving the productivity ever further toward zero.

Total cost owing a mess is huge and seldom leads to transformation of a functional application to a prototype. This leads to additional expenses in redeveloping the entire application. Alright, let's blame the managers for setting unrealistic deadlines, right? - No. Manager, marketers and the leadership team looks at us for the information they need to make promises and commitments. Its our job to tell them what we think about the proposed schedule. You might think of getting fired or setting a bad reputation, but tell me one manager that is ready to accept bad code just to meet the deadlines. No one is willing to take that risk as they know the challenges awaiting on deploying the code in the production environment. 




\end{document}